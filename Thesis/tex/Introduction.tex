\chapter*{Introduction}
\addcontentsline{toc}{chapter}{Introduction}

Reinforcement learning is a subfield of machine learning that aims to train agents to make decisions that will maximize a reward signal \citep{RLSuttonBarto}. This approach has been widely applied in the field of artificial intelligence, particularly in the context of training agents to play games. In a game setting, an agent's actions can be evaluated based on their impact on the agent's score or likelihood of winning. Through the process of reinforcement learning, the agent learns to make strategic decisions that maximize its reward by receiving positive reinforcement for good moves and negative reinforcement for suboptimal moves. This allows the agent to adapt and improve its performance over time as it plays the game. Research on reinforcement learning in games has demonstrated its effectiveness in a variety of contexts, including board games, video games, and real-time strategy games.

In the field of artificial intelligence, games can be classified as either continuous or discrete based on the nature of the action space and state space. Continuous games have a continuous action space, meaning that the possible actions an agent can take are not limited to a fixed set of options, but can vary continuously within a certain range. In contrast, discrete games have a discrete action space, meaning that the possible actions are limited to a fixed set of options.
Continuous games are often characterized by a high-dimensional state space, as they may involve a large number of variables that describe the game state. Discrete games, on the other hand, typically have a lower-dimensional state space, as the number of possible states is limited by the discrete action space.
In general, continuous games are more challenging to model and solve than discrete games, as they require more complex decision-making algorithms and may require more computational resources.

In this thesis, we will investigate the application of reinforcement learning to train agents to play a continuous 3D tunnel game, which I designed and implemented myself for this thesis work. The continuous game environment will be discretized into a set of states, and different reinforcement learning algorithms will be applied to train agents to play the game. The goal of this study is to determine whether it is possible for any of the agents to win the whole game, and to compare the performance of different agents that use different reinforcement learning algorithms.

The results of this study will contribute to the understanding of the potential of reinforcement learning for training agents to use discrete algorithms in a naturally continuous environment, and to provide insight into the strengths and weaknesses of different reinforcement learning algorithms in this context.\