%%% The main file. It contains definitions of basic parameters and includes all other parts.

%% Settings for single-side (simplex) printing
% Margins: left 40mm, right 25mm, top and bottom 25mm
% (but beware, LaTeX adds 1in implicitly)
\documentclass[12pt,a4paper]{report}
\setlength\textwidth{145mm}
\setlength\textheight{247mm}
\setlength\oddsidemargin{15mm}
\setlength\evensidemargin{15mm}
\setlength\topmargin{0mm}
\setlength\headsep{0mm}
\setlength\headheight{0mm}
% \openright makes the following text appear on a right-hand page
\let\openright=\clearpage

%% Settings for two-sided (duplex) printing
% \documentclass[12pt,a4paper,twoside,openright]{report}
% \setlength\textwidth{145mm}
% \setlength\textheight{247mm}
% \setlength\oddsidemargin{14.2mm}
% \setlength\evensidemargin{0mm}
% \setlength\topmargin{0mm}
% \setlength\headsep{0mm}
% \setlength\headheight{0mm}
% \let\openright=\cleardoublepage

%% Generate PDF/A-2u
\usepackage[a-2u]{pdfx}

%% Character encoding: usually latin2, cp1250 or utf8:
\usepackage[utf8]{inputenc}

%% Prefer Latin Modern fonts
\usepackage{lmodern}

%% Further useful packages (included in most LaTeX distributions)
\usepackage{amsmath}        % extensions for typesetting of math
\usepackage{amsfonts}       % math fonts
\usepackage{amsthm}         % theorems, definitions, etc.
\usepackage{bbding}         % various symbols (squares, asterisks, scissors, ...)
\usepackage{bm}             % boldface symbols (\bm)
\usepackage{graphicx}       % embedding of pictures
\usepackage{fancyvrb}       % improved verbatim environment
\usepackage{natbib}         % citation style AUTHOR (YEAR), or AUTHOR [NUMBER]
\usepackage[nottoc]{tocbibind} % makes sure that bibliography and the lists
			    % of figures/tables are included in the table
			    % of contents
\usepackage{dcolumn}        % improved alignment of table columns
\usepackage{booktabs}       % improved horizontal lines in tables
\usepackage{paralist}       % improved enumerate and itemize
\usepackage{xcolor}         % typesetting in color

%%% Basic information on the thesis

% Thesis title in English (exactly as in the formal assignment)
\def\ThesisTitle{Playing a 3D Tunnel Game Using Reinforcement Learning}

% Author of the thesis
\def\ThesisAuthor{Una Adilović}

% Year when the thesis is submitted
\def\YearSubmitted{2023}

% Name of the department or institute, where the work was officially assigned
% (according to the Organizational Structure of MFF UK in English,
% or a full name of a department outside MFF)
\def\Department{Department of Software and Computer Science Education (KSVI)}

% Is it a department (katedra), or an institute (ústav)?
\def\DeptType{Department}

% Thesis supervisor: name, surname and titles
\def\Supervisor{Adam Dingle, M.Sc.}

% Supervisor's department (again according to Organizational structure of MFF)
\def\SupervisorsDepartment{Department of Software and Computer Science Education (KSVI)}

% Study programme and specialization
\def\StudyProgramme{Computer Science}
\def\StudyBranch{Artificial Intelligence}

% An optional dedication: you can thank whomever you wish (your supervisor,
% consultant, a person who lent the software, etc.)
\def\Dedication{%
I would like to express my heartfelt appreciation to my supervisor, Adam Dingle, for his invaluable guidance and support throughout my academic journey, beginning with our first Programming 1 class and continuing through to the completion of this project. His patience and assistance were greatly appreciated and played a significant role in the success of this work.

In addition, I am deeply thankful to my mother for her unwavering belief in me and to my grandparents who made it possible for me to be here. Their contributions are greatly appreciated and will never be forgotten.
}

% Abstract (recommended length around 80-200 words; this is not a copy of your thesis assignment!)
\def\Abstract{%
Tunnel games are a type of 3D video game in which the player moves through a tunnel and tries to avoid obstacles by rotating around the axis of the tunnel. These games often involve fast-paced gameplay and require quick reflexes and spatial awareness to navigate through the tunnel successfully. The aim of this thesis is to explore the representation of a tunnel game in a discrete manner and to compare various reinforcement learning algorithms in this context. The objective is to evaluate the performance of these algorithms in a game setting and identify potential strengths and limitations. The results of this study may offer insights on the application of discrete tabular methods in the development of AI agents for other continuous games. (ADD RESULT)
}

% 3 to 5 keywords (recommended), each enclosed in curly braces
\def\Keywords{%
{tunnel game, reinforcement learning, artificial intelligence, algorithms}
}

%% The hyperref package for clickable links in PDF and also for storing
%% metadata to PDF (including the table of contents).
%% Most settings are pre-set by the pdfx package.
\hypersetup{unicode}
\hypersetup{breaklinks=true}

% Definitions of macros (see description inside)
\include{macros}

% Title page and various mandatory informational pages
\begin{document}
\include{title}

%%% A page with automatically generated table of contents of the bachelor thesis

\tableofcontents

%%% Each chapter is kept in a separate file
\chapter*{Introduction}
\addcontentsline{toc}{chapter}{Introduction}

Reinforcement learning is a subfield of machine learning that aims to train agents to make decisions that will maximize a reward signal \citep{RLSuttonBarto}. This approach has been widely applied in the field of artificial intelligence, particularly in the context of training agents to play games. In a game setting, an agent's actions can be evaluated based on their impact on the agent's score or likelihood of winning. Through the process of reinforcement learning, the agent learns to make strategic decisions that maximize its reward by receiving positive reinforcement for good moves and negative reinforcement for suboptimal moves. This allows the agent to adapt and improve its performance over time as it plays the game. Research on reinforcement learning in games has demonstrated its effectiveness in a variety of contexts, including board games, video games, and real-time strategy games.

In the field of artificial intelligence, games can be classified as either continuous or discrete based on the nature of the action space and state space. Continuous games have a continuous action space, meaning that the possible actions an agent can take are not limited to a fixed set of options, but can vary continuously within a certain range. In contrast, discrete games have a discrete action space, meaning that the possible actions are limited to a fixed set of options.
Continuous games are often characterized by a high-dimensional state space, as they may involve a large number of variables that describe the game state. Discrete games, on the other hand, typically have a lower-dimensional state space, as the number of possible states is limited by the discrete action space.
In general, continuous games are more challenging to model and solve than discrete games, as they require more complex decision-making algorithms and may require more computational resources.

In this thesis, we will investigate the application of reinforcement learning to train agents to play a continuous 3D tunnel game, which I designed and implemented myself for this thesis work. The continuous game environment will be discretized into a set of states, and different reinforcement learning algorithms will be applied to train agents to play the game. The goal of this study is to determine whether it is possible for any of the agents to win the whole game, and to compare the performance of different agents that use different reinforcement learning algorithms.

The results of this study will contribute to the understanding of the potential of reinforcement learning for training agents to use discrete algorithms in a naturally continuous environment, and to provide insight into the strengths and weaknesses of different reinforcement learning algorithms in this context.\
\chapter{Game Design}
The game that will be analysed is called "Space-run" and it involves attempting to accumulate the highest score possible by navigating through three distinct tunnels while avoiding various obstacles. The game is endless in nature, as the speed increases each time the player successfully completes all three tunnels.

\section{Player and Movement}
In "Space-run" the player assumes control of a character named Hans, who is responsible for dodging obstacles by moving left and right. In addition to these lateral movements, Hans also has the ability to shoot bullets and defeat certain in-game creatures, which results in a higher score. The player must utilize these abilities in order to progress through the game and achieve a high score.

\section{Obstacles}
As previously stated, the player must navigate through various obstacles in the game. These obstacles can be divided into three distinct categories, and each tunnel contains a unique subset of them. In the subsequent sections, we will delve deeper into these categories in order to better understand the challenges faced by the player.

\subsection{Traps}
\begin{figure}[h]
    \centering
    \includegraphics[width=\textwidth]{traps}
    \caption{Trap examples}
    \label{fig:mesh1}
\end{figure}
As depicted in Figure \ref{fig:mesh1}, a selection of the various trap types that the character Hans must avoid is presented. These traps, of which there are a total of 10, vary in their level of difficulty and can be either static or animated. These traps can be encountered in any of the three tunnels, and if the player fails to successfully evade them, they result in an instant death.

\subsection{Bugs}
\label{Bugs}
\begin{figure}[h]
    \centering
    \includegraphics[width=\textwidth]{bugs}
    \caption{Bugs}
    \label{fig:mesh2}
\end{figure}
In addition to traps, the game also features bugs as an obstacle (as shown in Figure \ref{fig:mesh2}). These bugs typically appear in the second tunnel and are designed to align with the player, making them more challenging to evade. However, they can still be avoided by the player. If the player chooses to engage with the bugs, they can be defeated by shooting three bullets at them. If the player collides with a bug, Hans will lose 25\% of his battery life(for more information on battery life, see Section \ref{AdditionalFeatures}).

\subsection{Viruses}
	\begin{figure}[h]
    \centering
    \includegraphics[width=\textwidth]{viruses}
    \caption{Bacteriophage and Rotavirus}
    \label{fig:mesh3}
\end{figure}
The third and final type of obstacle in the game are viruses (illustrated in Figure \ref{fig:mesh3}). These viruses are typically found in the third tunnel and, like bugs, can be eliminated through the use of three bullets. They also align with the player's movement, making them more difficult to evade, although it is still possible to do so. Bacteriophage, a subtype of virus, will result in an instant death if the player comes into contact with them. Rotaviruses, on the other hand, will cause the player's character to become sick for a brief period of time. During this illness, it is crucial for the player to avoid coming into contact with another Rotavirus, as this will result in the end of the game.

\section{Additional Features}
\label{AdditionalFeatures}
There are several other features of the game that are worth mentioning. One of the most significant of these is the battery life of the player's character, Hans, which is displayed on the right side of the screen. As Hans is designed to resemble a computer, it is necessary for him to recharge his battery throughout the game by collecting energy tokens. This will fully restore his battery capacity. There are three main ways in which Hans can lose battery life: running causes a constant reduction of 1\%, each bullet shot costs 1\% of the battery life, and coming into contact with a bug results in a reduction of 25\% (as described in Section \ref{Bugs}). If the battery reaches 0\%, Hans will die and the game will end.

Finally, it should be noted that upon successfully navigating through all three types of tunnels, the game will increase in speed and the player will once again encounter the same tunnels, looping through them indefinitely until the player loses.

\section{Score Count and Winning}
The score of the game is based on the length of time that the player is able to survive. Additionally, each time a player successfully shoots down a bug or virus, their score increases by 10 points. As previously mentioned, the game is designed to be played indefinitely, but for the purpose of this study, we have set the game to be considered won after an agent successfully completes nine tunnels, reaching level 10.

\chapter{Implementation of the Game}
``Space-run'' was developed using the Godot Engine (version v3.2.3.stable) , an open-source game engine licensed under the MIT license. It is a cross-platform tool that offers a range of features for game development, including a visual scripting language, 2D and 3D graphics support, and a powerful physics engine. The Godot Engine utilizes a node-based architecture, where nodes are organized within scenes that can be reused, instanced, inherited, and nested. This structure allows for efficient project management and development within the engine. The game was written entirely in GDScript, the primary scripting language of the Godot Engine (\cite{GodotDocs}).

In addition to using the Godot Engine, I also utilized Blender (version 6.2.0) \cite{blender} for creating and animating the characters in the game. Blender is a popular open-source 3D modeling and animation software that offers a range of features for creating detailed and realistic characters. The characters were then imported into the Godot Engine using the .glTF 2.0 \cite{gltf} file format, which is a widely supported file format for exchanging 3D graphics data.

The source code for the game, and for the whole thesis can be found at the following repository \cite{spacerunai}.

\section{The top-level organization}
\begin{figure}[h]
    \centering
    \includegraphics[width=0.8\textwidth]{game_tree}
    \caption{Structure of Game.tscn}
    \label{fig:game_tree}
\end{figure}

The main scene for the game, referred to as \texttt{Game.tscn}, is depicted in Figure \ref{fig:game_tree}. It includes several nodes, including Ground, UI, Sound, Game, Hans, and Tunnels. The Ground node is a CSGBox\footnote{CSGBox is a 3D object that represents a box with a Constructive Solid Geometry (CSG) shape.} that serves as the ground in the game, while the UI and Sounds nodes handle the user interface and audio aspects, respectively. The Game, Hans, and Tunnels nodes contain the majority of the game's functionality. Specifically, the Game node manages the overall gameplay, the Hans node controls the player character, and the Tunnels node manages the movement and appearance of the tunnels.

For a more in-depth understanding, let us examine some of the core aspects of the game in the following sections.

\section{Game}
The script for the Game node is the initial point of the game session and includes both the \texttt{\_start()} and \texttt{\_game\_over()} methods. It also serves as a link between the game and the agent environment described in Chapter \ref{agent_code_chapter}, and as such includes all of the necessary set methods for the agent environment. These methods allow for communication between the game and the agent environment, enabling the agent to interact with and influence the game.

The following text describes the core functionalities of the main methods within \texttt{Game.gd}:
\begin{itemize}
\item The ready function is called at the start of the game's execution and, after setting up the environment, it triggers the start function. This function initiates the gameplay and sets the necessary conditions for the game to proceed.
\item As described in more detail in Chapter \ref{agent_code_chapter}, the user can specify environment parameters and a starting level for the agent through the command line. These parameters determine the obstacles that the player will face and the starting position of the player character, Hans. The start function incorporates these parameters into the obstacle arrays and positions Hans accordingly. The function also generates the obstacles for the designated starting level. The creation and deletion of obstacles during gameplay is discussed in Section \ref{tunnel_script} of this chapter.
\item The game over function manages the end of the game and sends a signal to the top-level script, \texttt{Main.gd} (described in Chapter \ref{agent_code_chapter}), indicating that the game has ended. It also provides \texttt{Main.gd} with the necessary information about the game's status and outcome.
\end{itemize}

\section{Hans}
The next node we want to examine is Hans. While \texttt{Hans.tscn} is a scene with the main character and its necessary animations, what interests us more is the \texttt{Hans.gd} and its key components.

\begin{lstlisting}
func _physics_process(delta): 
	# code used to update the variables and UI
		...
			
    tunnels.delete_obstacle_until_x(curr_tunnel,translation.x - tunnels_children[curr_tunnel].translation.x + 50)
    
    # create a trap in the next tunnel every 50 meters
    if translation.x < new_trap:
        create_new_trap()
    
    # updating score 
    score._on_Meter_Passed()
    
    # Hans's movement
    var velocity = Vector3.LEFT * speed
    velocity = move_and_slide(velocity)
    
    # in case Hans colided with somehting, handle it properly
    check_collisions()
    
    # bugs and viruses need to move torwards Hans    
    tunnels.bug_virus_movement(delta, curr_tunnel)
    
    if isShootingButtonPressed:
        shoot()
        
    # type needs to be calculated before 
    # so we know where the next trap is on x axis
    var type = calc_type()
    state.update_state(calc_dist(),calc_rot(),type)
\end{lstlisting}

The primary function within \texttt{Hans.gd} is the \texttt{\_physics\_process()}, which is called on every tick of the game. It handles the main aspects of the player character through the use of various methods and functions. These include deleting passed obstacles, creating new obstacles every 50 meters, updating the score, handling the movement of the player character, bugs and viruses, and determining the current state of the player. The state label, which is displayed on the upper right corner of the screen (as shown in Figure \ref{fig:third_tunnel}), is the primary information that agents receive when making decisions about their next move, as described in Chapter \ref{agent_code_chapter}. The \texttt{\_physics\_process()} function also handles collisions and shooting if the player chooses to do so. Overall, this function plays a crucial role in the gameplay and management of the player character.

It is also worth noting that this script handles the movement of the tunnels to the back as Hans passes them, with the first tunnel being moved to be after the third one. This feature allows for the game to be infinite, as the tunnels are constantly cycled and reused.

\begin{figure}[h]
    \centering
    \includegraphics[width=\textwidth]{third_tunnel}
    \caption{State}
    \label{fig:third_tunnel}
\end{figure}

\section{Tunnels}
\label{tunnel_script}
The Tunnels node, which is a child of the main scene in the game tree, contains three child nodes of the Spatial type\footnote{Spatial node is a type of node that represents a 3D object or transformation in the game world. It is a versatile node that can be used to create and manipulate 3D objects, including meshes, materials, and lighting. Spatial nodes are often used as the root node for 3D objects in a scene, and they can be nested inside other Spatial nodes to create hierarchical transformations.} (level1, level2, and level3) and each of these nodes includes a CSGTorus\footnote{ CSGTorus node is a type of 3D object that represents a torus shape in the game world.} node, which represents the physical appearance of the tunnels. Obstacles are added to the appropriate level node as instances. The \texttt{Tunnels.gd} script, which is attached to the Tunnels node, handles many of the previously mentioned functions such as obstacle creation and deletion and tunnel rotation. In the following code snippets, we will examine the \texttt{Tunnels.gd} script in greater detail.

\begin{lstlisting}
func _physics_process(delta):   
	# gets move from the agent
	# in case we chose the Keyboard agent 
	#this will just return input from the keyboard
    var move = game.agent.move(game.state.get_state(), game.score.get_score(), game.num_of_ticks)
    
    #rotates the tunnel
    if move[0] == 1:
        var tunnel = get_child(hans.get_current_tunnel())
        tunnel.rotate_object_local(Vector3.RIGHT,-ROTATE_SPEED * delta)
    elif move[0] == -1:
        var tunnel = get_child(hans.get_current_tunnel())
        tunnel.rotate_object_local(Vector3.LEFT,-ROTATE_SPEED * delta)
        
    # shoot if necesarry
    if not hans == null: # if it is not instanced we can't call the function       
        hans.switch_animation(move[1] == 1)
\end{lstlisting}

The \texttt{\_physics\_process()} function within the \texttt{Tunnels.gd} script serves as the primary connection between the agent and the game. As shown in the provided code, the function retrieves the next move from the agent and rotates the tunnel accordingly, potentially including shooting as well.

\begin{lstlisting} 
func create_first_level_traps(tunnel):
    level <- level we are making traps for
    num_of_traps <- randomly pick number of traps to be added
    x <- x position of the first trap
    
    for n in num_of_traps:
        x <- update x to next position
        if x is outside the tunnel:
            break
        # create one trap at x
        create_one_obstacle(level, x) 
\end{lstlisting}

The function depicted in the code above serves to generate obstacles in the starting tunnel. By periodically creating traps in the tunnel ahead, the game is able to prevent lag caused by an excessive number of objects existing simultaneously. For that reason, this function is used only once, at the beginning of the game.

\begin{lstlisting} 
func create_one_obstacle(level,x):
    scene <- pick which kind of obstacle will be added
    tunnel <- get the level we are making traps for
    i <- randomly pick an obstacle
    
    using previous information, make an instance of the obstacle
    set its position at x and rotate it randomly by n * 60 degrees  
\end{lstlisting}

The tunnels are positioned along the x axis, and this function allows for the creation of obstacles within them at specific x positions and a random rotation.

\begin{lstlisting}           
func delete_obstacle_until_x(level,x):
    tunnel <- get current tunnel

    for child in children of the tunnel:
        if child is an obstacle:
            if we passed the child:
                remove it
            else:
                return
\end{lstlisting}

As previously mentioned, by dynamically deleting passed obstacles, the game is able to maintain a stable performance and avoid overloading the system. The provided code demonstrates the implementation of this function.

\chapter{Applied Algorithms}
The algorithms utilized in this thesis fall under the categories of Monte Carlo methods and Temporal Difference (TD) Learning. These methods involve the selection of actions by an agent over time in order to maximize a reward signal. The agents are unaware of the environment dynamics of the game and are only provided with information regarding the current state and score. In this chapter, we will discuss the specific algorithms implemented in the project. Firstly, however, a few key concepts need to be introduced:

A \textbf{policy} is a mapping from states to actions that determines the actions an agent takes in each state. An \textbf{$\epsilon$-greedy policy} is a type of policy in which the agent takes the action that maximizes the expected reward with probability (1 - $\epsilon$), and takes a random action with probability $\epsilon$. A \textbf{state} is a representation of the current situation of the environment, and an \textbf{action} is a decision made by the agent in response to the current state. A \textbf{reward} is a scalar value that represents the immediate feedback received by the agent for its action in a given state.

In the previous chapter, the representation of the state in the game was discussed, and it was briefly mentioned that the actions an agent can take are [\texttt{left, right, forward, left + shoot, right + shoot, and forward + shoot}].\\ As for the reward, the most natural choice is the score variable from the game.

Before looking into individual algorithms, there is one more key concept to introduce: \textbf{exploration vs exploitation}. In the field of reinforcement learning, the exploration-exploitation trade-off refers to the balancing act between discovering new information or strategies and utilizing existing knowledge to maximize reward. Exploration involves trying out different actions or strategies in order to gather more information about the environment and its rewards, while exploitation involves utilizing the information gathered to maximize reward. Finding the right balance between exploration and exploitation is crucial in reinforcement learning, as excessive exploration can lead to suboptimal results, while excessive exploitation can result in suboptimal performance.

\section{Monte Carlo}
The Monte Carlo method is a type of reinforcement learning algorithm that uses a sample of the past experiences of the agent to estimate the value function. It can be used with either on-policy or off-policy learning, depending on how the samples are collected. On-policy learning means that the agent is using the same behaviour policy to collect samples as it is using to evaluate the value function. First visit Monte Carlo is a variant of on-policy Monte Carlo that only considers the first time a state is visited in an episode, as opposed to all visits. Optimistic initial values are a technique used in Monte Carlo methods to encourage exploration by setting the initial value of all states to a high number, encouraging the agent to visit as many states as possible in order to learn their true value. For the purpose of this thesis, we have used on-policy first visit Monte Carlo algorithm, with both initial optimistic value and an $\epsilon$-greedy policy as tools to enforce exploration. 


\begin{figure}[h]
    \centering
    \begin{lstlisting}
    For each episode:
    		Initialize value function for all states to a high number (optimistic initial value).
		Initialize the epsilon-greedy policy with a probability of exploration (epsilon).
		Loop for each step of episode, t = T-1, T-2,..., 0:
			Initialize state and action
			Take the action and observe the next state and reward.
			If the state has not been visited before in this episode (first visit):
				update the value function using the reward.
			Choose the next action using the epsilon-greedy policy.
		
		Update the epsilon-greedy policy based on the results of the learning process.
	\end{lstlisting}	
    \caption{On-policy first-visit Monte Carlo control (for $\epsilon$-greedy policy)}
    \label{algo:MC}
\end{figure}

\section{Temporal Difference Learning}
Temporal difference (TD) learning is a type of reinforcement learning algorithm that estimates the value function using the difference between the immediate reward and the expected future reward. It can be used with either on-policy or off-policy learning, depending on how the samples are collected.
On the other hand, Monte Carlo methods require a complete episode to finish before the value function can be updated, whereas TD learning can update the value function incrementally as the agent takes actions. This means that TD learning can start learning and adapting to the environment much earlier than Monte Carlo methods.

\begin{figure}[h]
    \centering
    \begin{lstlisting}
    Loop through each episode:
    		Initialize state S
    		For each step of the episode
    			Pick an action A using epsilon-greedy policy
    			Take action A and observe the reward R and the next state S' 
    			Update the policy depending on the algorithm
    			S <- S'
    		Until S is terminal
	\end{lstlisting}
    \caption{Tabular TD learning}
    \label{algo:TD}
\end{figure}

The main difference between SARSA, Q-learning, expected SARSA, and double Q-learning is the way in which they estimate the value of the action-value function. The pseudo code above shows us the common core idea for all the TD learning algorithms used in this thesis. Now, we can look at them individually, based on their update functions.
\begin{figure}[h]
    \centering
   $$Q(S,A) \leftarrow Q(S,A) + \alpha [R + \gamma Q(S',A') - Q(S,A)$$
    \caption{SARSA update function}
    \label{algo:S}
\end{figure}
SARSA stands for "state-action-reward-state-action", and is an on-policy TD learning algorithm. This means that it uses the same behavior policy to collect samples as it is using to evaluate the action-value function. In SARSA, the action-value function is updated using the current state, current action, reward, next state, and next action.
\begin{figure}[h]
    \centering
    $$Q(S,A) \leftarrow Q(S,A) + \alpha [R + \gamma {max}_a Q(S',a) - Q(S,A)]$$
    \caption{Q-learning update function}
    \label{algo:QL}
\end{figure}
Q-learning is an off-policy TD learning algorithm, meaning that it uses a different behavior policy to collect samples than it is using to evaluate the action-value function. In Q-learning, the action-value function is updated using the current state, current action, reward, and next state, with the next action chosen using the greedy policy.
\begin{figure}[h]
    \centering
    $$Q(S,A) \leftarrow Q(S,A) + \alpha [R + \sum_a \pi (a|S') Q(S',a) - Q(S,A)]$$
    \caption{Expected SARSA update function}
    \label{algo:ES}
\end{figure}
Expected SARSA is a variant of SARSA that uses the expected value of the next action, rather than the actual next action, to update the action-value function. This allows the algorithm to take into account the probability of each possible next action, rather than just the action chosen by the behavior policy.
\begin{figure}[h]
    With 0.5 probability:
    $$Q_1(S,A) \leftarrow Q_1(S,A) + \alpha [R + \gamma  Q_2(S', argmax_a Q_1(S',a)) - Q_1(S,A)]$$
    	Else:
    	$$Q_2(S,A) \leftarrow Q_2(S,A) + \alpha [R + \gamma Q_1(S', argmax_a Q_2(S',a)) - Q_2(S,A)]$$
    \caption{Double Q-learning update function}
    \label{algo:DQL}
\end{figure}
Double Q-learning is a variant of Q-learning that uses two action-value functions to estimate the value of each action. The two functions are updated independently, and the final action-value estimate is the average of the two estimates. This helps to reduce the bias introduced by the greedy policy used in Q-learning.
\chapter{Implementation of the Agents}
\label{implOfAgents}
In this project, the RL agents are designed such that their functionality is encapsulated within a top-level scene called \texttt{Main.tscn}. Within this scene, there is an instance of the selected agent, and the code makes use of several functions implemented by the agent in order to interact with the environment. Specifically, the required functions are: \texttt{move()}, \texttt{init()}, \texttt{start\_game()}, \texttt{end\_game()}, \texttt{save()}, \texttt{set\_seed\_val()}, \texttt{get\_and\_set\_agent\_specific\_parameters()}, and \texttt{get\_n()}.

The purpose of most of these functions is self-explanatory. \texttt{init()} and \texttt{save()} are used to initialize and save the agent's internal state, respectively, and are called only once per experiment. \texttt{start\_game()} and \texttt{end\_game()} are called at the beginning and end of each episode, while \texttt{move()} is called by the \texttt{Tunnels.gd} script, and it is in this function that the agent makes a decision about which action to take based on the current state and score. The remaining functions pertain to the internal structure of the agent and are not relevant to the reader.

\section{Hierarchy}
\begin{figure}[h]
    \centering
    \includegraphics[width=0.8\textwidth]{agents_tree}
    \caption{Agents hierarchy inside the project}
    \label{fig:agents_tree}
\end{figure}

In the current design of the game, there are a total of 8 agents implemented, 5 of which utilize some form of reinforcement learning (RL) algorithm. These RL agents share a common superclass called ``Learning Agent'', while 3 of them are further subclassed under the ``TD Agent'' class (see Figure \ref{fig:agents_tree}). As previously discussed, the RL algorithms can be broadly divided into two categories: Monte Carlo methods and temporal difference (TD) learning. The TD algorithms differ only in their update function, and so it was deemed appropriate to group them under the same superclass. However, the Double Q-Learning agent, which utilizes separate policies and requires additional modifications, was implemented as a separate subclass or the ``Learning Agent''. The implementation details of these agents will be further elaborated upon in the subsequent sections. 

Before proceeding further, it is crucial to emphasize that, as previously mentioned when describing the state in Chapter 3, the agent has the ability to perform movements in various directions, including moving right, forward, or left, and each of these movements can be performed in combination with shooting. Therefore, on each time step, the move function of the agent returns a list of two elements: the first element signifies the movement direction, where a value of -1 represents left, 0 represents forward, and 1 represents right; while the second element determines whether the agent will shoot or not, with a value of 1 indicating yes and 0 indicating no for shooting.


\section{Simple Agents}
To facilitate testing of the game environment, several simple agents were implemented. These agents serve as baseline models and are used to ensure that the environment is functioning as intended before more sophisticated RL agents are developed. There are three simple agents in total: a ``Keyboard agent'' that receives input from the player via the keyboard, a ``Static agent'' that always chooses the forward action without shooting, and a ``Random agent'' that chooses a random action at each time step. 

\section{Learning Agent}
The Learning Agent class serves as a base class for all the reinforcement learning agents in this project. It provides a set of shared functions and features that are used by all agents, such as reading and writing data to a file and debugging statements. In terms of decision-making, these agents all follow an $\epsilon$-greedy policy, whereby they select the action with the highest value for a given state with a certain probability, or randomly choose any action with the remaining probability. Each of the subclasses of the Learning Agent class then implements specific code that is unique to that particular agent.

\begin{algorithm}
\caption{Choosing an action using $\epsilon$-greedy policy}
\begin{algorithmic}[1]
\Function{choose\_action}{action}
    \State $\textit{epsilon\_action} \gets \textit{false}$
    \If{not $\textit{is\_eval\_game}$} \Comment{Used for continunous evaluation}
        \State $\textit{epsilon\_action} \gets rand.\text{randf\_range}(0,1) < EPSILON$
        \If{$\textit{epsilon\_action}$}
            \State $action \gets ACTIONS [rand.\text{randi\_range}(0,len(ACTIONS) - 1)]$  
        \EndIf
    \EndIf
    \State \Return $action$
\EndFunction
\end{algorithmic}
\end{algorithm}

\subsection{Common parameters and behaviours}
In regards to the present implementation of the reinforcement learning algorithms for the 3D tunnel game, certain aspects of the game learning step and parameter implementation are unique to this project and merit discussion. For instance, we should look into the learning step in the game. The agent will not request a new move until the state has changed, despite the fact that it may seem more natural for a new decision to be made on every game tick. This has the effect of reducing the number of decisions that the agent must make in a given episode, but also results in intriguing policy behaviours that are elaborated upon in Chapter \ref{experiments_chapter}.

Furthermore, there are several parameters that are common to all learning agents, some of which were briefly discussed in the preceding chapter. In this section, we will examine in more detail how these parameters were integrated into this particular project. The initial optimistic value parameter is the simplest one to explain, as it is implemented in a straightforward manner. In particular, each time a state-action pair is added to the policy, its value is set to a predetermined number. This number (\texttt{initOptVal}), along with other parameters mentioned subsequently in this subsection, are part of the sub-options agent's parameter that is specified through the command line (see \ref{commOpt}).

The following two parameters worth noting are \texttt{eps} and \texttt{epsFinal}, which are responsible for the random moves executed by the $\epsilon$-greedy policy. These parameters allow the user to specify the starting and ending values of $\epsilon$, which are then suitably decreased after each game played. The formula for this decrease is as follows: $$decrease = (\frac{epsFinal}{eps})^{\frac{1.0}{n}}$$

Here, \texttt{n} represents the number of games being played. The reason behind this epsilon decrease is to change the ratio between exploration and exploitation over time. At the beginning of the experiment, the \texttt{eps} value is higher, and thus random moves happen more often, causing the agent to try actions it would otherwise oversee. Later, when the policy is a bit stabilized, the \texttt{eps} value becomes smaller so it would allow the agent to play longer games and possibly win (if the \texttt{eps} value was high throughout the whole experiment, the agent would have a bigger chance of choosing an inadequate move and thus untimely ending the game).

Finally, it is pertinent to discuss the discounting value gamma (defined by \texttt{gam}). In this project, discounting is used in the following manner: $$\gamma^{next\_step.time - curr\_step.time}$$

Normally, the gamma value grows by one power on each step. However, as previously stated, learning steps in this implementation do not occur on every tick, but instead occur when the state changes. As a result, they may vary in size. To avoid uneven discounting, the time is calculated for each new decision made by the agent using the formula provided:  $$(game.num\_of\_ticks * 33) / 1000.0$$

\subsection{Monte Carlo Agent}

\begin{algorithm}
\caption{Updating policy for Monte Carlo}\label{mcUpdate}
\begin{algorithmic}[1]
\State abs
\end{algorithmic}
\end{algorithm}

The Monte Carlo method is a type of reinforcement learning algorithm that updates its policy only after an episode is completed. This is done by iterating through the entire episode and increasing the number of visits and total return for each state-action pair, if this is their first visit inside this episode. The total return is calculated using the formula shown in the code snippet \ref{mcUpdate}, while the number of visits is simply incremented by 1. To determine the optimal action, the agent compares the ratio of total return to number of visits for each possible action at a given state. This calculation is performed at each state transition during the episode \footnote{Instead of calling the move function periodically, the agents will always choose the same action based on the current state. Only once the state has changed, the new action is chosen based on the accumulated score and the new state.}.
As previously mentioned, the $\gamma$ variable in the equation shown in \ref{mcUpdate} serves as a discount factor, meaning that the last move made, which resulted in the agent's death, will receive the highest penalty. As we move further down the list of moves, their significance decreases. It is important to note that if the value of $\gamma$ is set to 1, all moves are given equal weight. 

\subsection{TD Agent}	
Unlike the Monte Carlo methods, which update their policies only after the completion of an episode, TD agents update their policies in real-time, after each action is taken. To accomplish this, all TD agents have a shared function called \texttt{move()}, which call the function displayed in \ref{tdUpdate}. 

\begin{algorithm}
\caption{Updating policy for TD Agent}\label{tdUpdate}
\begin{algorithmic}[1]
\State abs
\end{algorithmic}
\end{algorithm}


The update of the policy for a specific state value in TD learning involves multiple variables, most of which have been previously discussed. One new variable is $\alpha$, which represents the total number of visits for a given state action pair divided by one. Furthermore, to make this calculation, a variable uniquely computed by each TD algorithm, known as \texttt{new\_state\_action}, is required. Various methods for computing this variable can be found in Figure \ref{tdIndivUpdate}. If the current state is a terminal state, the \texttt{new\_state\_action} will not be utilized and instead, it will be replaced with a value of 0, as indicated in the update code.

\begin{algorithm}
\caption{Update functions for different algorithms}\label{tdIndivUpdate}
\begin{algorithmic}
\State abs
\end{algorithmic}
\end{algorithm}

In SARSA, the \texttt{new\_state\_val} is calculated based on the value of the next action the agent will take, denoted as \texttt{new\_action}. On the other hand, Q-learning uses the value of the best action possible in the next state, denoted as \texttt{best\_action}. These two variables are equal if greedy policy is implemented. However if we consider $\epsilon$-greedy policy, then they might differ based on weather a random action has been chosen. Expected SARSA combines these two approaches by taking the expected value of all possible actions in the next state. 

\begin{algorithm}
\caption{Update functions for Double Q-Learning}\label{dqlUpdate}
\begin{algorithmic}
\State abs
\end{algorithmic}
\end{algorithm}

Similar to the agents in the TD class, the update for the Double Q-Learning agent occurs each time the agent changes its state. The update process is slightly different. In this method, two separate action-value functions, denoted as Q1 and Q2, are used to estimate the maximum action value for a given state. At each update step, one of the Q-values is selected randomly and updated using the other Q-value as a reference. This process helps to reduce the overestimation of action values and leads to more stable learning.
\chapter{Experiments}
\label{experiments_chapter}
In this chapter, we will evaluate the performance of the various agents when confronted with different combinations of obstacles, as well as presenting interesting observations made during the experiments. One of the key questions we aim to answer is whether any of the agents are capable of learning to play the entire game.

\section{Individual traps environments}
This section discusses the performance of agents in environments containing only a single type of trap, and no other obstacles such as bugs or viruses.

\begin{figure}[h]
    \centering
    \includegraphics[width=0.9\textwidth]{all_traps}
    \caption{Performance of all agents in single-trap environments}
    \label{fig:all_traps}
\end{figure}

In Figure \ref{fig:all_traps}, for each type of trap, there is a plot with \texttt{eps=0.2} and \texttt{initOptVal=20} and how each agent performed in an environment containing only that trap type\footnote{For the purpose of having all of these plots in one figure, they were manually modified. However, the calculations are still the same as described in the previous chapter.}. These values were picked because with most traps the agents performed well under these conditions.

It should be noted that all plots within this section have smoothing applied with \texttt{window=10}. Thus certain spikes may not be visible. Additionally, unless otherwise suggested, you can assume that the values that were produced are an average of 5 different seeds. The aim is to show a realistic picture on how the agent would perform, and not show the occurrences in which the outcome was satisfactory but rather account for the failures in reproducing the perfect policy as well.

In some cases, of course, the hyperparameters used in the experiments in Figure \ref{fig:all_traps} were not ideal. However, for most trap/agent pairs, we managed to find at least one combination of hyperparameters where the agent found an optimal or a policy that won some games but not consecutively, across multiple seeds. The only exceptions were the \texttt{MonteCarlo} agent with the \texttt{Hex} trap, \texttt{DoubleQLearning} with the \texttt{HexO}, \texttt{Triangles}, and \texttt{X} traps, and \texttt{SARSA} and \texttt{QLearning} with the \texttt{HexO} trap. \texttt{ExpectedSARSA} was the only agent that produced a good policy on multiple occasions for all individual trap types and even performed exceptionally well with certain hyperparameters for the \texttt{Hex}, \texttt{I}, \texttt{MovingI}, \texttt{O}, and \texttt{Walls} traps, in which cases it found an optimal policy across multiple seed values.

\begin{figure}[h]
    \centering
    \includegraphics[width=\textwidth]{balls}
    \caption{\texttt{Balls} trap experiments}
    \label{fig:balls_eg}
\end{figure}

Choices of the hyperparameters are a very important factor. Most experiments described in this section used the hyperparameter values \texttt{eps=0.2} and \texttt{initOptVal=20.0}\\ or \texttt{initOptVal=100.0}. A good example of how much hyperparameters can influence the outcome is visible in Figure \ref{fig:balls_eg}. Performance in the \texttt{Balls} trap environment varies significantly based on the initial optimistic value used. This plot underscores the importance of carefully selecting hyperparameters for reinforcement learning.

\begin{figure}[h]
    \centering
    \includegraphics[width=\textwidth]{i}
    \caption{\texttt{I} trap experiments}
    \label{fig:i_eg}
\end{figure}

\begin{figure}[h]
    \centering
    \includegraphics[width=\textwidth]{movingi}
    \caption{\texttt{MovingI} trap experiments}
    \label{fig:movingi_eg}
\end{figure}

In some cases the aforementioned variations in hyperparameter selection lead to highly desirable outcomes. This is exemplified by the results presented in Figures \ref{fig:i_eg} and \ref{fig:movingi_eg}, which demonstrate the efficacy of the \texttt{ExpectedSARSA} agent. Notably, on the right hand side of the both plots, the \texttt{ExpectedSARSA} agent was able to achieve optimal performance early on in the game with all seed values. The \texttt{ExpectedSARSA} agent, in a very large number of experiments, has outperformed its counterparts, sometimes by a significant amount. This is particularly evident when considering its performance in simpler environments such as traps \texttt{I} and \texttt{MovingI}, where it is apparent that the agent is capable of learning extremely well. While other agents have performed well on these specific traps as well, their success may not be immediately apparent from the averaged results depicted in the plots.

Although performing these experiments with a larger number of seeds would ideally yield even more accurate results, we hope that the picture we presented provides a reasonable representation of the agents' performance in the demonstrated environments.

\section{Traps environment}
\begin{figure}[h]
    \centering
    \includegraphics[width=\textwidth]{allTraps}
    \caption{All traps examples}
    \label{fig:alltraps_eg}
\end{figure}

This section delves into the exploration of a highly intricate environment of all traps combined, which is the most complex one barring the full game. To clarify, the \texttt{Traps} environment contains all 10 trap types mentioned in the previous section, and omits any bugs, virus or token type of obstacles. As depicted in Figure \ref{fig:alltraps_eg}, the majority of agents were unable to perform optimally in this environment. \texttt{ExpectedSARSA} was the only agent able to learn an optimal policy for almost all random seeds tested, as evidenced by its score nearing 500 in the right plot\footnote{The plots for this environment have been smoothed with \texttt{window=100}.}, which was the approximate winning score value in all experiments. Moreover, \texttt{ExpectedSARSA} not only managed to learn an optimal policy once but did so with different hyperparameters and seed values on multiple occasions, leading to winning streaks of 30 games and early termination of the experiment. This outcome is the most favourable for any environment. The figure displays the averaged value of 9 seeds for all agents under the specified parameters. 

The experiments conducted for this environment were systematic and involved matching commonly used \texttt{eps} and \texttt{initOptVal} to test if the agents could learn. In these experiments, the difference in learning between the \texttt{ExpectedSARSA} agent and the others is even more pronounced. However, on the left plot visible in Figure \ref{fig:alltraps_eg}, where \texttt{eps=0.4} and \texttt{initOptVal=20}, the \texttt{MonteCarlo} agent performed reasonably well, attaining an average score of approximately 100, which is substantially superior to the other agents, except for \texttt{ExpectedSARSA}.

\begin{figure}[h]
    \centering
    \includegraphics[width=0.8\textwidth]{allTraps_table}
    \caption{All traps with \texttt{ExpectedSARSA}, \texttt{MonteCarlo} and \texttt{QLearning} agents}
    \label{fig:traps_table_eg}
\end{figure}

When training in the all-traps environment we set the \texttt{rots} parameter to 22.  That's because this environment contains the \texttt{Hex} and \texttt{Walls} traps, which require a minimum of 22 rotations for learning to be feasible at all. This means that our experiments in this environment had many more states than in any single-trap environment. Considering that in this case we have 10 different trap types, there are 220 states with \texttt{Traps} environment. For that reason we picked \texttt{n=2500} for all of the experiments in this section. That's because in our experiments we've generally found that learning is most successful when the number of episodes is at least 10 times the number of states. As a result of having this many \texttt{rots} values, with some simpler traps, there can be many safe rotations that the agent can choose from. For that reason, going forward could be viable in multiple adjacent states, when the next trap ahead requires for example \texttt{rots=6} when trained individually.

The table in Figure \ref{fig:traps_table_eg} provides a comparison of policies from three different experiments, each reproduced with only one seed value, that yielded a policy that managed to win enough times to have an average score between 100 and 200 or more. The purpose is to see how far off the agents were from an optimal policy. The blue rows represent the \texttt{ExpectedSARSA} agent, in an experiment performed with \texttt{eps=0.2} and \texttt{initOptVal=100}. The agent stopped an experiment early, after 437 games with initial number of games (\texttt{n}) being 2500. Considering it won 30 games in a row, each one lasting 15 levels, we can assume that the learned policy is optimal for this environment. The red rows represent the \texttt{MonteCarlo} agent, which performed reasonably well with \texttt{eps=0.2} and \texttt{initOptVal=20}, winning 282/2500 games, but the experiment did not terminate early, suggesting a suboptimal policy. The pink rows represent the \texttt{QLearning} agent, which won 142/2500 games with \texttt{eps=0.4} and \texttt{initOptVal=20}. It should be noted that the combination of a seed value and hyperparameters that yielded the best results were picked for the \texttt{MonteCarlo} and \texttt{QLearning} agents, and in other cases, they won fewer or no games under this environment. Lastly it should be noted that, \texttt{SARSA} and \texttt{DoubleQLearning}  performed poorly, with average scores in all experiments under all hyperparameter combinations, being not more than 20.

Multiple instances in the data show the phenomenon discussed in Section \ref{intbeh} of this chapter. For instance, upon closer examination of rotation values 54 and 71 in the rows pertaining to the Balls trap, it becomes evident that the \texttt{ExpectedSARSA} agent opted to alternate between those two rotation types, whereas the other two agents, \texttt{Monte Carlo} and \texttt{QLearning}, chose to proceed using only one or both of the rotations. This trend can be observed in several other cases within the data, and it is highly probable that, with so many rotation options available, any of the three methods would lead to the agent safely passing the trap. Nevertheless, it is a fact that \texttt{ExpectedSARSA} learned a better policy than \texttt{MonteCarlo} and \texttt{QLearning}. However, for certain traps, all or at least two of the agents had satisfactory policies (such as the \texttt{Hex} trap), whereas for others, \texttt{MonteCarlo} and/or \texttt{QLearning} were observed taking actions that could not be deemed optimal when compared to the \texttt{ExpectedSARSA} agent. An example of such an instance can be found in rotation values 275 and 292 with trap \texttt{X}, where the \texttt{QLearning} agent attempted to switch between the two rotations to pass, while both \texttt{ExpectedSARSA} and \texttt{MonteCarlo} avoided it, suggesting that remaining in that rotation was not safe and that the agent should try to move to another rotation in a timely manner.

\section{Tokens environment}
\begin{figure}[h]
    \centering
    \includegraphics[width=0.6\textwidth]{tokensEnv}
    \caption{\texttt{Tokens} example}
    \label{fig:tokens}
\end{figure}

The \texttt{Tokens} environment is characterized by its simplicity, as it lacks obstacles that pose a lethal threat to the agent. The agent's task is to collect tokens at regular intervals to prevent its battery from draining completely. The number of rotations needed in this environment is six, making it relatively easy to train. Figure \ref{fig:tokens} illustrates that the average performance of all agents is commendable, even when \texttt{n=20}\footnote{Note that no smoothing was applied to this plot.}. In subsequent sections, we will delve into more intriguing findings when tokens are incorporated into a larger environment, and explore their impact on the behaviour of the agents.

\section{Individual Bugs and Viruses environments}
\begin{figure}[h]
    \centering
    \includegraphics[width=0.9\textwidth]{bugs_ind}
    \caption{\texttt{LadybugFlying}, \texttt{LadybugWalking} and \texttt{Worm} environments examples}
    \label{fig:bugs_ind_eg}
\end{figure}

This section depicts individual performance of bug and virus type obstacles. They include three (\texttt{Worm, LadybugWalking, LadybugFlying}) and two (\texttt{Bacteriophage, Rotavirus}) obstacles, respectively, and are different from environments that consist solely of traps. One key difference is that the agent can also use its ability to shoot in \texttt{Bugs} and \texttt{Viruses} environments. This means that, for the first time in this chapter, our agents have 6 actions to choose from, and we aim to analyse how this affects their behaviour.

We begin by showcasing the performance of all agents in environments containing only a single type of bug or virus. For this purpose, we have evaluated their performance on 50 games, each with \texttt{eps=0.2} and \texttt{initialOptVal=100.0}. We chose these parameters as they resulted in the best overall performance of the agents. The plots presenting all agents in this section are an average of five different seeds, and each plot has been smoothed with a \texttt{window=10}. The left-hand side of both figures displays the agents' performance when shooting was not enabled, whereas the right-hand side depicts the scores when the agents could shoot.

\begin{figure}[h]
    \centering
    \includegraphics[width=0.8\textwidth]{bugs_ind_mc}
    \caption{\texttt{LadybugWalking} with \texttt{MonteCarlo} agent examples}
    \label{fig:bugs_ind_mc_eg}
\end{figure}

The category of obstacles referred to as bugs in the game environment behaves differently from viruses or traps. This is due to the fact that Hans only loses energy upon contact with any of the bug type obstacles, and only when the battery life is depleted to 0\% does the game terminate, or when the game is won. The performance of some agents in this environment is lower than in others, as illustrated in Figure \ref{fig:bugs_ind_eg}. However, the \texttt{MonteCarlo} agent consistently performs well, particularly with the \texttt{LadybugWalking} obstacle. Conversely, the \texttt{QLearning} agent appears to perform poorly in this environment, suggesting that it may not be the most suitable method for these seemingly inconsistent bug obstacles. As battery life is not part of the state value, the ideal behaviour for the agent would be to either always shoot (if permitted) or always avoid the these obstacles. This behaviour is precisely what is observed in Figure \ref{fig:bugs_ind_mc_eg}, where the \texttt{MonteCarlo} agent does not shoot at all in the left plot (suggesting that its avoiding the obstacles), while in the right one, it shoots in almost all actions. Although the left plot may not have achieved an optimal policy, it has derived one that allows the agent to win at least 50\% of the time.

\begin{figure}[h]
    \centering
    \includegraphics[width=0.8\textwidth]{viruses_ind}
    \caption{\texttt{Bacteriophage} and \texttt{Rotavirus} environments examples}
    \label{fig:viruses_ind_eg}
\end{figure}

Figure \ref{fig:viruses_ind_eg} displays the performance of the agents for each virus type obstacle. Notably, \texttt{ExpectedSARSA} exhibited a high level of performance in both cases where shooting was not enabled. However, in the case of \texttt{Rotavirus} when the shooting was involved, it performed the worst out of all the agents. It is possible that the agent was confused by the \texttt{Rotavirus} behaviour, since when Hans hits a \texttt{Rotavirus} for the first time, he becomes sick, and if it is hit again during his sick period, it results in his death. In contrast, the behaviour of \texttt{Bacteriophage} is more akin to the traps, as it kills Hans on the spot upon contact.

\begin{figure}[h]
    \centering
    \includegraphics[width=0.8\textwidth]{viruses_ind_dql}
    \caption{\texttt{Rotavirus} environment with \texttt{DoubleQLearning} agent examples}
    \label{fig:viruses_ind_dql_eg}
\end{figure}

To diverge a little from the performance of the agents all together, let's take a look at Figure \ref{fig:viruses_ind_dql_eg} which shows two instances of the \texttt{DoubleQLearning} agent with the \texttt{Rotavirus} obstacle and \texttt{shooting=enabled}. Each plot used only one seed value and the smoothing was not applied. In both of the cases in this figure, the agent learned an optimal policy and thus terminated the game early. However, for the left plot, even though there are some actions in which the agent chooses to shoot, it doesn't actually shoot down any obstacles. This is evident by the fact that the winning score is only around 500, which is the minimum winning score achieved after 15 levels.  This might explain why in these conditions, the \texttt{DoubleQLearning} agent did not perform as well as some of the others in the overall performance analysis (Figure \ref{fig:viruses_ind_eg}). On the other hand, the right plot shows a policy in which that the agent learned to shoot down the \texttt{Rotavirus} obstacles and achieved a much higher winning score, demonstrating the importance of properly utilizing the shooting action when it is enabled.

\section{Bugs and Viruses environments}
\begin{figure}[h]
    \centering
    \includegraphics[width=0.9\textwidth]{BV800ns}
    \caption{\texttt{Bugs} and \texttt{Viruses} environments examples}
    \label{fig:bv800ns_eg}
\end{figure}

The current section compares the performance of agents in the full \texttt{Bugs} or \texttt{Viruses} environments and their combination with \texttt{Tokens} when run with the same hyperparameters. Plots that display the performance of all agents in this section are averaged over 10 seeds and smoothed with \texttt{window=100}. In Figure \ref{fig:bv800ns_eg} we see the average performance of the agents when shooting is not allowed. Similar to the traps environment, the \texttt{ExpectedSARSA} agent exhibited the best performance. On the other hand, when faced with the full \texttt{Bugs} environment, \texttt{QLearning} agent showed similar behaviour to that seen on individual bugs obstacles and performed worse than the other agents.

\begin{figure}[h]
    \centering
    \includegraphics[width=0.8\textwidth]{bbt_vvt}
    \caption{\texttt{Bugs}, \texttt{Viruses} and their combination with \texttt{Tokens} examples}
    \label{fig:bbt_vvt_eg}
\end{figure}

Figure \ref{fig:bbt_vvt_eg} shows plots where shooting actions are available to the agent. Here, the \texttt{MonteCarlo} agent had the best performance in all four cases. An interesting result is that when faced with \texttt{env=[Bugs,Tokens]}, this agent performed better than in the only \texttt{Bugs} environment, while with \texttt{env=[Viruses,Tokens]}, it performed worse than with \texttt{Viruses} alone. Considering how running into a bug influences Hans, adding \texttt{Tokens} to the \texttt{Bugs} environment was beneficial as long as the tokens were picked up when possible and the agent was conservative with shooting. In contrast, in the environment containing both \texttt{Viruses} and \texttt{Tokens}, the agent could still easily lose if Hans ran into \texttt{Bacteriophage} once or \texttt{Rotavirus} twice in a row. When having unlimited shooting in the \texttt{Viruses} environment alone, all agents performed much better than when \texttt{Tokens} were part of it as well.

Overall, in environments that include \texttt{Tokens}, the ratio of shooting and no shooting actions was approximately even. 

\begin{figure}[h]
    \centering
    \includegraphics[width=0.9\textwidth]{BV_02_100_1}
    \caption{\texttt{Bugs} and \texttt{Viruses} with and without shooting comparison}
    \label{fig:BV_02_100_1_eg}
\end{figure}


In Figure \ref{fig:BV_02_100_1_eg}, the performance of \texttt{MonteCarlo} agent shown environments is displayed for specific hyperparameters and with the same seed value (smoothing window value is 100). For each environment, two plots were created, one where the agent is allowed to shoot and another where shooting is disabled. The two plots depicting \texttt{Bugs} exhibit noticeable differences, as the agent found an optimal policy when shooting was not available. With shooting enabled, even though the direction of the agent's movement for each state is quite similar to the ones in the first plot, it failed to win any games. In contrast, with the \texttt{Viruses} environment, the agent's movement is almost identical in both cases, and the plot shows that the agent performed similarly well, with the right plot showing an average and winning scores approximately 10 times higher than on the left, as a consequence of the ability to shoot and due to that gain higher scores.

\section{Full game environment}
In this section, we explore the performance of the agents in the most challenging environment, the full game. The results of these experiments are not surprising, given the complexity of the environment. In the figures presented in this section, we averaged the results of 10 seeds, and each plot was smoothed with a window of 100. The plots on the left side show the performance of the agents in the environment where shooting is disabled, while the right side ones show the environment where shooting is enabled. The right side was trained on 2000 more games than the left one considering that the number of actions possible increased.

\begin{figure}[h]
    \centering
    \includegraphics[width=0.8\textwidth]{full_game}
    \caption{Full game example}
    \label{fig:full_game_eg}
\end{figure}

Looking at Figure \ref{fig:full_game_eg}, we can see that in the environment without shooting, \texttt{ExpectedSARSA} performed the best, as expected. The scores of \texttt{MonteCarlo} and \texttt{QLearning} agents were not too bad, considering the complexity of the environment and the fact that \texttt{QLearning} agent underperformed in the \texttt{Bugs} environment. On the right plot, \texttt{ExpectedSARSA} is still in the lead compared to the other agents. However, considering that the agents were allowed to shoot in this case, the scores did not improve significantly.

\begin{figure}[h]
    \centering
    \includegraphics[width=0.8\textwidth]{full_game_cf}
    \caption{Catastrophic forgetting example}
    \label{fig:full_game_cf_eg}
\end{figure}

In Figure \ref{fig:full_game_cf_eg} we can observe an occurrence that has not been discussed before but is present throughout our experiments, namely the concept of catastrophic forgetting \cite{russell2010artificial}. As the name suggests, this is the notion that the agent learns a good policy and due to further exploration of the environment, the policy changes to something suboptimal. Ideally, the agent would come back to the previous policy, but more often than not, this is not the case. In the plot on the right, we can see that this is exactly what happened to the \texttt{ExpectedSARSA} agent during this experiment. The policy it learned in the first couple of thousand games was far from optimal, but it achieved a better score than the policies used after the sudden drop at approximately 2500 games.

\begin{figure}[h]
    \centering
    \includegraphics[width=0.7\textwidth]{full_game_es_win}
    \caption{Full game with \texttt{ExpectedSARSA} example}
    \label{fig:full_game_es_eg}
\end{figure}

The highly anticipated encounter detailed in Figure \ref{fig:full_game_es_eg} showcases a noteworthy outcome, as it portrays the \texttt{ExpectedSARSA} agent learning to play a full game by discovering an optimal policy under specific parameter settings and only one seed value. It is important to highlight that this experiment was conducted while the winning level was 10 and not 15, thus the winning score is lower than in the other examples shown in this chapter. As part of the figure we can see the policy the agent acquired. This example serves as an illustration of how reinforcement learning agents are capable of learning to play this game when given the right conditions. However, it is important to note that this was a singular occurrence, and no similar outcomes were observed when shooting actions were introduced to the experiments. It is reasonable to conclude that while it may not be impossible for the agent to learn such a policy again, it may necessitate obtaining a fortuitous seed value.

\section{Interesting behaviours}
\label{intbeh}
\begin{figure}[h]
    \centering
    \includegraphics[width=\textwidth]{discountingExample}
    \caption{Discounting example}
    \label{fig:discounting_eg}
\end{figure}

In this chapter, we aim to discuss certain unexpected findings that surfaced during our experimentation. One of the immediate observations can be seen in Figure \ref{fig:discounting_eg}\footnote{No smoothing was applied to any of the plots in this subsection.}. We conducted experiments on two distinct environments, \texttt{env=[I]} and \texttt{env=[X]}, and for each environment, we carried out experiments with discount rates of \texttt{gam=0.85} and \texttt{gam=1.0} for all agents. As evident from the plots, the lower gamma value exhibited considerably poorer performance than when no discounting (\texttt{gam=1.0}) was applied. This trend is not limited to these specific environments and testing conditions but rather observed consistently across all our experimentation. This result is counterintuitive since it seems logical that penalizing the last action more than previous ones would result in a better policy.

\begin{figure}[h]
    \centering
    \includegraphics[width=\textwidth]{discountingExplanation}
    \caption{Discounting explanation}
    \label{fig:discounting_expl}
\end{figure}

Upon further investigation, we discovered that in some cases, a lost game for the agent does not result from the last action directly but rather from a chain reaction initiated by a previous bad decision. As seen in Figure \ref{fig:discounting_expl}, the agent's last action of going right at rotation \texttt{360} to reach a safe one, \texttt{60}, is not a poor decision in itself but rather the best possible action in that state\footnote{As confirmed by a human player, rotation 60 is safe for trap type I.}. However, analysing the last four actions taken by the agent, it becomes clear that it attempted to reach rotation \texttt{60} by going right from the rotation \texttt{180}. With a high score of \texttt{451.7} (Figure \ref{fig:discounting_eg}), the agent undeniably had high speed value, making it move forward very quickly, and while this policy may have been effective in the early game before the agent attained its current speed\footnote{It should recalled that after every 3 levels the agent's speed increases.}, there is simply not enough time for the agent to rotate at this point in the game. In this case, one could argue that the fourth action from the last was responsible for the agent's loss. For cases like this, we believe that the agent performs better when all actions are penalized equally, i.e. when \texttt{gam=1.0}.

\begin{figure}[h]
    \centering
    \includegraphics[width=\textwidth]{trianglesExample}
    \caption{\texttt{Triangles} example}
    \label{fig:triangles_intbeh_eg}
\end{figure}

Another intriguing behaviour emerged as a result of our learning architecture. As previously noted, we chose to have the agent not take a new action every time its \texttt{move()} function is called, but rather return the same action until the state changes. This resulted in a substantial reduction in the number of different actions taken by the agent, given that the move function is invoked every game tick, while state changes occur less frequently. This approach led to a behaviour that could not have been predicted at the outset. Figure \ref{fig:triangles_intbeh_eg} provides an illustration of this behaviour (with some sentiment \texttt{env=[Triangles]} was chosen as this was the first environment on which the behaviour was noticed).

Ordinarily, \texttt{env=[Triangles]} does not offer any safe rotations unless \texttt{rots=7} or more. As shown in the plot, the agent will certainly learn with this rotation value. However, if the agent is given only \texttt{6} rotation values to choose from, it will develop a policy that rapidly oscillates between two rotations, keeping the player character, Hans, on the edge of those rotations, allowing him to safely pass through the trap. This behaviour is not confined to situations where the agent is ``forced'' to make such a decision. During training, in many instances, the agent will learn to stay on the edge rather than advance into a completely safe rotation. There does not appear to be a preference for one or the other; rather, the policy the agent discovers first is determined by other experimental factors. It can be concluded that this behaviour arose solely because the agent did not alter its action until the state changed. In my opinion, this discovery is one of the most exciting outcomes of this project.
\chapter*{Conclusion}
\addcontentsline{toc}{chapter}{Conclusion}

In conclusion, this thesis has investigated the efficacy of different reinforcement learning algorithms in various environments in a game implemented in the Godot game engine. The results showed that the \texttt{ExpectedSARSA} algorithm performed moderately or exedingly well in all environments, while the performance of other algorithms varied. In particular, \texttt{MonteCarlo} demonstrated impressive results in environments featuring bug and virus obstacles. All algorithms displayed adequate performance in environments with individual trap types, while performance in environments with multiple obstacle types was not as consistent. The results underscore the importance of the random actions that the agent receives during training and the balance between exploration and exploitation in reinforcement learning. Future research could explore ways to influence these random actions to potentially achieve better outcomes.

%%% Bibliography
\include{bibliography}

%%% Figures used in the thesis (consider if this is needed)
\listoffigures

%%% Tables used in the thesis (consider if this is needed)
%%% In mathematical theses, it could be better to move the list of tables to the beginning of the thesis.
\listoftables

%%% Abbreviations used in the thesis, if any, including their explanation
%%% In mathematical theses, it could be better to move the list of abbreviations to the beginning of the thesis.
\chapwithtoc{List of Abbreviations}

%%% Attachments to the bachelor thesis, if any. Each attachment must be
%%% referred to at least once from the text of the thesis. Attachments
%%% are numbered.
%%%
%%% The printed version should preferably contain attachments, which can be
%%% read (additional tables and charts, supplementary text, examples of
%%% program output, etc.). The electronic version is more suited for attachments
%%% which will likely be used in an electronic form rather than read (program
%%% source code, data files, interactive charts, etc.). Electronic attachments
%%% should be uploaded to SIS and optionally also included in the thesis on a~CD/DVD.
%%% Allowed file formats are specified in provision of the rector no. 72/2017.
\appendix
\chapter{Attachments}

\section{First Attachment}

\openright
\end{document}
